% openany makes chapters start on even pages too
\documentclass[openany, 12pt]{book}

% sets margins of text inside page to 1 inch
\usepackage[top=1in, bottom=1in, left=1in, right=1in]{geometry}

% this will indent the first paragraph in a chapter
\usepackage{indentfirst}

% to use href
\usepackage{hyperref}

% to use images
\usepackage{graphicx}
% where images are stored
\graphicspath{{images/}}

\title{
  Introduction to Fitness \\
  \vskip 0.5cm
  \small A Technical Book on Fitness}
\author{Mihail Dunaev}
% this will remove date
\date{}

\begin{document}
  \maketitle
  \tableofcontents

  \chapter{Introduction}
  
	I have no background in fitness or nutrition. In school I studied Computer Science and now I work as a Software Engineer. What made me get more 
	into fitness was my weight: I was fat. Multiple times in my life. First, growing up as a kid I was fat and I managed to lose the weight in 
	secondary school just by eating less. Second time I got fat was in university, just because I lacked any motivation to study for exams and would
	power my way through with food and energy drinks. After uni I found it much harder to lose weight than I previously knew. I would follow a lot 
	of diets (keto\footnote{\href{https://en.wikipedia.org/wiki/Ketogenic_diet}{Ketogenic Diet on Wikipedia}}, food replacement like 
	soylent\footnote{\href{https://en.wikipedia.org/wiki/Soylent_(meal_replacement)}{Soylent on Wikipedia}} or 
	huel\footnote{\href{https://en.wikipedia.org/wiki/Huel}{Huel on Wikipedia}}), sometimes going really low in calories intake, feeling like I 
	have no energy and gaining the weight back on after ending the diet. I needed a better solution to this: I decided to start going to the gym.
	
	I guess my goal was to build muscle and lose fat. I've always liked muscles as a kid, just never got the time to look into it and start
	working out. This was my chance. I had a lot of questions: how do I train? What do I eat? Are there other things I need to pay attention to?
	One solution would be to hire a personal trainer to help me achieve this goal. What stopped me from doing this is the way my brain works: I knew
	I would just be given a list of things to do without any explanation. I really like to understand how things work and I would end up being 
	disappointed. This is also a big part of my life by now, I definitely wanted to understand well how everything works. I started looking into fitness
	the same way I look into everything: start searching on google, looking at professional bodybuilders, what they say, does it make sense etc. 
	I feel that if you want to know a subject really well you should look for competitions related to that subject, and see what the people involved
	have to say about it. This is why I started following people that compete in bodybuilding shows, strongman competitions and so on. 	
	What bothered me was how poorly organised the fitness information I found was. I couldn't find a single place that could take me from 0 knowledge
	to getting started in a matter of few hours. I had to watch a lot of youtube videos, read a lot of articles and posts in fitness communities until
	things were clear in my head. Now that I know all this information I think it's possible to put it all in one place: this is the purpose of this 
	book, to get you started with your fitness journey, especially if your goal is to build muscles and lose fat. And as I understand, this is the case
	for	most people working out.
	
	I just want to stress out: there is nothing wrong with being fat. If you get really fat it is unhealthy and you'll end up with 
	health problems. However, the main reason I don't want to be fat is because I get anxiety from it. I feel like crap, especially if I take my shirt 
	off in public. Saying I don't care about it is just lying to myself and I try my best to not do that, just for my mental health. This anxiety is
	something I can't control so for me the only solution would be to stay in shape. Besides, I already said I like muscles, so becoming muscular would
	make me feel proud of myself.
	
	The book is structured in two parts: nutrition \& workout. There is a lot of information in here, you don't necessarily need to understand all of it
	to get going. That's why at the end I just added an example of everything I did to get in my current shape without extra explanations. I will try 
	my best to present information in an unbiased way, presenting what people think works and what not, what I tried on myself etc. If you think 
	something is wrong or don't agree with some of the information presented here, this is an open source book so feel free to submit a 
	PR!\footnote{\href{https://docs.github.com/en/free-pro-team@latest/github/collaborating-with-issues-and-pull-requests/about-pull-requests}
	{About Pull Requests on Github Docs}} At the end of the day I am a practical person, I only believe in results and what works in real life. 
	I can say that the information I describe in this book worked really well on me, as you can see in the picture below.
	
	\begin{figure}[h]
		\centering
		\includegraphics[scale=0.2]{transformation.jpg}
		\caption{From fat to muscular in 7 years: my lifelong struggle with being fat}
		\label{fig1}
	\end{figure}
	
	
  \chapter{Nutrition}
  
  	\section{Calories, BMR, TDEE}
  	
	I feel like I need to explain how food works first before anything else. Your body is an energy convertor. It gets energy from food\footnote{\href{https://en.wikipedia.org/wiki/Food_energy}{Food Energy on Wikipedia}} and converts it to heat, movement (kinetic energy) and 
	electrical energy for thinking. The amount of energy needed without movement (so for heat, thinking and perhaps others things too) is referred 
	to as basal metabolic rate or BMR\footnote{\href{https://en.wikipedia.org/wiki/Basal_metabolic_rate}{Basal Metabolic Rate on Wikipedia}} and it
	doesn't change much from day to day. If you include the energy for movement too you get your total daily energy expenditure or TDEE. If you eat
	more than your TDEE in a day, the extra food will be stored on your body either as fat or muscle. If you eat less then your body will have to go
	to fat stores and muscles and break them down to get the extra food energy you need. Energy is measured in $kcal$, but most of the time you will
	only see the term calorie with the same meaning (basically $kcal$ is the scientific term which was replaced by calorie when it started being used
	by food industries). To put energy values into perspective, the average human would need 2,000 calories for heat every day, or so I've seen in a
	physics course a long time ago. If I run on the treadmill for 1h I get a message that I burned roughly 600 calories. 1 Big Mac has almost 600 calories.
        It's important to understand that knowing exactly how many calories a meal has is next to impossible. There will always be small differences in every
        ingredient you use. For example not all loafs of bread are the same size, not all strawberries contain the same amount of sugar and so on. It's also
        impossible to know exactly how much energy you burn in a day. However, estimates work really well in practice. As long as you eat the same meals every week,
        you will either lose, gain or maintain weight.
	
	People have tried to come up with formulas to compute BMR from different factors, such as height, age, sex or body fat percentage (this is just
	the proportion of fat you have in your body relative to your whole mass, so 100 $\times$ fat mass / body mass). At first only mass (m), height (h) 
	and age were taken into account in Harris-Benedict formula for BMR from 1919\footnote{\href{
	https://en.wikipedia.org/wiki/Harris\%E2\%80\%93Benedict_equation}{Harris-Benedict on Wikipedia}}
	\begin{equation}
		BMR = 13.7516m + 5.0033h + 66.4730
	\end{equation}
  	This formula was later revised in 1984 with just a few minor changes. Later in 1990 Mifflin St Jeor\footnote{\href{https://pubmed.ncbi.nlm.nih.gov/2305711/}{A new predictive equation for resting energy expenditure in healthy individuals (1990)}} came with 2 formulas for BMR, one for men (\ref{males}) and one for women (\ref{females}).
	\begin{equation}
		\label{males}
		BMR (males) = 10m + 6.25h - 5a + 5
	\end{equation}
	\begin{equation}
		\label{females}
		BMR (females) = 10m + 6.25h - 5a - 161
	\end{equation}
	Finally Katch-McArdle\footnote{\href{https://books.google.co.uk/books/about/Essentials_of_Exercise_Physiology.html?id=L4aZIDbmV3oC}{Essentials of Exercise Physiology Book by Katch \& McArdle (2006)}} included body fat percentage (f) into the equation, removing age and height
	\begin{equation}
		BMR = 370 + (21.6m (1 - \frac{f}{100}))
	\end{equation}
	This is an interesting point because body fat percentage does affect how many calories you burn even at rest. The rule I know is that 10 pounds of muscle would burn 50 kcal in a day at rest, while 10 pounds of fat will only burn 20 kcal\footnote{\href{https://www.webmd.com/diet/obesity/features/8-ways-to-burn-calories-and-fight-fat}{Burning Calories on WebMD}} (less than half), so if you're 80kg with 15\% body fat you will burn more calories at rest than someone who is 80kg with 20\% body fat. This also explains the Mifflin St Jeor above, since women have naturally more fat than men. 
	
	Let's take an example using the last formula: if you weight $70kg$ and your body fat percentage is $18\%$ then your BMR should be $370 + (21.6 \times 70 \times (1 - 18/100)) = 1609.84$ calories. Add your movement energy to this and you get your TDEE. I haven't spent the time trying to derive how to
	compute this one (e.g. from kinetic energy) because all of these formulas are great but at the end of the day they are just for your orientation.
	The best way to compute your TDEE is to actually measure it: without changing your habits, eat 2,000 calories every day for 1-2 weeks. Weight
	yourself every day: does your weight change? If no then it's safe to assume your TDEE is 2,000 calories. Does your weight go up? It probably means
	your TDEE is lower. Keep adjusting your calories intake until you find your TDEE. 
	
	In theory you should be able to tell your TDEE without having
	to change your diet again just by looking at how much weight you gained / lost in the initial 1-2 weeks: you should lose 1lb (or 0.45kg) of mass at 
	a total deficit of 3500 calories\footnote{\href{https://www.ncbi.nlm.nih.gov/pmc/articles/PMC2376744/}{What is the Required Energy Deficit per 
	unit Weight Loss? (2008)}}. Let's take an example again: you ate 2,000 calories for 2 weeks. During these 2 weeks you gained 2.2lbs (or 1kg) on the
	scale. According to the 3500 rule, you were at a total surplus of $3500 \times 2.2 = 7700$ calories. This surplus was achieved in 14 days, so the
	surplus each day was 550 calories. This means your actual TDEE is 2,550 calories. However I found this rule to not work on me, trying to adjust 
	accordingly didn't put me at maintenance and I kept changing weight. As long as you always adjust to results you will be fine. I would personally
	start with an online TDEE calculator\footnote{\href{https://tdeecalculator.net/}{Example of online TDEE calculator}} (there are plenty out there) 
	just to get a value to work with, then keep adjusting intake until I hit my maintenance.
	
	
	\section{Body Fat Percentage}
	
	Body fat percentage is discussed a lot in fitness because it affects how you look. Fat is something that is stored on your body between skin and muscles. The more fat you have on you the less visible your muscles will be. However, just taking the absolute value of fat mass is not a good indicator of how well your muscles are showing, since taller people will have more fat mass but more body surface to spread it across. So instead we can look at the proportion of fat in relation to muscles. The body fat percentage will normally dictate certain features you can see on your body, for example the average guy will have his abdominal muscles ("abs") showing around 10-14\% body fat\footnote{\href{https://www.healthline.com/health/body-fat-percentage-for-abs}{Healthline Article on Abs}}. To better understand what I'm talking about look at the image below that I found on builtlean.com\footnote{\href{https://www.builtlean.com/body-fat-percentage-men-women/}{Body Fat Percentage Photos Of Men \& Women}}. I cannot confirm the numbers to be correct but they give a good indication in my opinion to what body fat percentage looks like at different values (note that lower body fat percentage will not give you bigger muscles, it just happens to be the case in the pictures). Usually people say that an ideal body fat \% (both in terms of health and looks) is around 12\% for men and around 24\% for women. Another thing to keep in 
	mind is that your genetics will influence how low in body fat \% you can get. For some men reaching 5\% might be impossible without taking steroids for example. Also going
	under 10\% is usually not considered healthy anymore, a lot of people complain about mood, sleep and even sex drive problems once you are that low in body fat \%\footnote{\href{https://www.youtube.com/watch?v=IHvmtvzOfDg}{VitruvianPhysique on Optimal Body Fat \% on Youtube}}. 
	
	\begin{figure}[h]
		\centering
		\includegraphics[scale=0.6]{bodyfatpercentage.jpg}
		\caption{Different body fat \% for both men and women}
	\end{figure}
	
	Computing body fat percentage is not easy. There is no 100\% accurate way of doing it. You can take pictures of yourself in the mirror and then compare with the images above, a lot of people estimate this way and I find it good as well. If you want a more accurate way of doing it though, there are a few options out there. The most accurate way is an MRI scan\footnote{\href{https://pubmed.ncbi.nlm.nih.gov/9655763/}{Cadaver validation of skeletal muscle measurement by magnetic resonance imaging and computerized tomography}}. However this is not available to the public as far as I know. This leaves us with the second most accurate option I know, which is a DEXA scan\footnote{\href{https://en.wikipedia.org/wiki/Dual-energy_X-ray_absorptiometry}{DEXA on Wikipedia}}. This is a machine that does an x-ray scan of your body. It shows quite some details, for example the lean mass and fat mass in your arms, trunk (core) and legs. You can see an example of a DEXA scan result below.
	\begin{figure}[h]
		\centering
		\includegraphics[scale=0.5]{dexa-scan.png}
		\caption{Result of a DEXA scan, showing body fat for different body parts}
	\end{figure}
However, DEXA scans can have errors too, and a lot of people talk against it\footnote{\href{https://www.youtube.com/watch?v=2Gg4Jm5KS1Y}{Greg Doucette on DEXA Scans on Youtube}}\footnote{\href{https://www.youtube.com/watch?v=P17bcpYE8Ew}{Brain Shaw on DEXA Scans on Youtube}}. The scan is also expensive, I did it in London at bodyscan for roughly \pounds 100.
	
	Before DEXA scans, hydrostatic weighing\footnote{\href{https://en.wikipedia.org/wiki/Hydrostatic_weighing}{Hydrostatic Weighing on Wikipedia}} was considered the most accurate method available to the public. You had to step on a scale underwater and the value you get helps estimate your body density, which can be used to approximate your body fat percentage.	Another way of estimating body fat percentage which is similar to hydrostatic weighing is whole-body air displacement plethysmography\footnote{\href{https://en.wikipedia.org/wiki/Air_displacement_plethysmography}{Air Displacement Plethysmography on Wikipedia}} (for example Bod Pod\footnote{\href{https://www.cosmed.com/en/products/body-composition/bod-pod}{Bod Pod}}).
	
	More common methods of estimating body fat are the skinfold methods\footnote{\href{https://en.wikipedia.org/wiki/Body_fat_percentage\#Anthropometric_methods}{Skinfold Methods on Wikipedia}} (using a device called caliper) and using bioelectrical impedance analysis\footnote{\href{https://en.wikipedia.org/wiki/Bioelectrical_impedance_analysis}{Bioelectrical Impedance Analysis on Wikipedia}}. The first method tries to determine the thickness of the fat layer under the skin. You use the caliper at various spots on the body and look up the measurement in a table which will tell you the estimated body fat. It's best if you use a personal trainer or someone with experience to perform the reading. For bioelectrical impedance analysis a small electric current is sent through the body and the resistance of the whole body is computed, which depends on body fat percentage. Some scales also state they can compute body fat percentage just from your weight, height, age and sex. However this is not really possible unless you don't lift at all and the extra weight comes from fat alone. Just to recap all the methods described in this section, from most accurate to least one:
	
	\begin{enumerate}
		\item Magnetic Resonance Imaging (MRI) and Computerized Tomography (CT)
		\item Dual-energy X-ray absorptiometry (DEXA / DXA)
		\item Hydrostatic weighing
		\item Whole-body air displacement plethysmography
		\item Skinfold methods (calipers)
		\item Bioelectrical impedance analysis
	\end{enumerate}
	
	\section{Macronutrients}
	
	The food you eat can be broken down into different components. The stuff that gives you energy (e.g. for movement) is called macronutrients --- these are carbohydrates
	(carbs), fat \& protein. The stuff that doesn't give you energy is called micronutrients (e.g. vitamins \& minerals). You still need micronutrients to be a healthy individual.
	The reason people talk so much about macronutrients (or macros) is because your body process them differently: carbs can be broken down and used for energy faster than fats,
	proteins are the only macros that can be stored as muscle on your body and so on. It's hard for you to tell how many carbs, fats and proteins the food you're eating contains.
	You have to read the label or research in advance. To give some example of macros in different types of food: bread is mostly carbs, in 100g of bread you have 49g of carbs,
	9g of protein and 3.2g of fat. This composition is also what tells us how many calories 100g of bread has. The rule is that 1g of carbs has 4 calories, 1g of fat has 9 calories
	and 1g of protein has 4 calories, just like carbs. If we add these values up for bread we get $49 \times 4 + 9 \times 4 + 3.2 \times 9 = 196 + 36 + 28.8 = 260.8$ calories. Your
	body can convert carbs and proteins into fat to store on your body as fuel for future days, but it cannot convert back fat to carbs or proteins. It's worth mentioning that
	fats and carbs cannot be converted to proteins either, so for building muscles you need proteins from food alone, since there's no way to store proteins on your body. 
	
	There are different theories that you should eat this many grams of carbs and this many grams of fat. One such example is the zone 
	diet\footnote{\href{https://www.healthline.com/nutrition/zone-diet}{The Zone Diet on Healthline}}. This says you should eat 40\% carbs, 30\% fat and 30\% protein. For example,
	if you need to eat 2,000 calories a day and want to follow the zone diet, you should aim to eat 145.5g of carbs (582kcal), 109g of fat (981kcal) and 109g of protein (436kcal).
	However, from my experience it doesn't really matter how you split carbs and fat, it's just a matter of preference. At the end of the day it's calories in and calories out that
	matters\footnote{\href{https://www.youtube.com/watch?v=ssmJ50HRTp8}{Greg Doucette on Macros on Youtube}}. You should pick a diet you feel comfortable with, so if you like carbs just eat more carbs, if you like fat just eat more fat.
	It does matter how much protein you have though, since it's the only thing your body can use for muscle growth. A common recommendation for building muscles is to eat
	1g of protein per pound of body weight, or 2.2g per kg\footnote{\href{https://www.healthline.com/nutrition/how-much-protein-per-day}{Protein Intake on Healthline}}. For example,
	if you weight 70kg you should eat 154g of protein every day.
	
	If you really want to try and follow a certain macro split, you might want to compute how many grams of carbs, fat and protein to consume based on their percentage. I know
	I had to compute this when I was trying to follow certain percentages. To make your life easier you can plug in your values into the formulas below. The value $c$ is the 
	percentage of carbs, $f$ is percentage of fat, $p$ is percentage of protein and $T$ is the total caloric intake:
	
	\begin{equation}
		Carbs(g) = c \times \frac{T}{4 \times c + 9 \times f + 4 \times p}
	\end{equation}
	
	\begin{equation}
		Fat(g) = f \times \frac{T}{4 \times c + 9 \times f + 4 \times p}
	\end{equation}
		
	\begin{equation}
		Protein(g) = p \times \frac{T}{4 \times c + 9 \times f + 4 \times p}
	\end{equation}
	
	In our previous example, $T = 2000$, $c = 40\% = 0.4$, $f = 30\% = 0.3$ and $p = 30\% = 0.3$. If we plug these values in the formulas above we get
	
	$$ \frac{T}{4 \times c + 9 \times f + 4 \times p} = \frac{2000}{4 \times 0.4 + 9 \times 0.3 + 4 \times 0.3} = 363.63 $$
	
	$$ Carbs = 363.63 \times 0.4 = 145.5g $$
	
	$$ Fats = 363.63 \times 0.3 = 109g $$

	$$ Protein = 363.63 \times 0.3 = 109g $$		
	
	\section{Fat Stores, Distribution and Carb Stores}
	
	If you eat more than you should in a day, so more than your TDEE, your body will store the excess food either as fat or muscle on your body. As I previously mentioned, only
	proteins can be stored as muscle and only if you train accordingly (more on this later). The way fat gets stored across your body (fat distribution) for example on arms, belly,
	hips etc, depends on genetics and you can't really influence it unless of course, you undergo surgery to remove fat cells from your body. For example some people store a lot of
	fat on hips, other store it on legs and bums, others on face and so on. One clear factor that influences fat distribution is gender, usually women will store fat on legs, bums,
	arms and breasts while men will store most fat around their belly. I have a pretty unfortunate fat distribution, since I store fat the same way women do, I get a lot on arms,
	legs, bum and even around breasts. I used to think this is due to hormonal imbalances (maybe too much estrogen and too low testosterone) but it turns out this is not the case,
	since my testosterone levels are really high and estrogen really low. From what I've read, the fat distribution is influenced by fat cell receptors: alpha-2 and beta receptors.
	The beta receptors will release fat while the alpha-2 will stall it, so the ratio of alpha-2 to beta receptors will dictate how you lose or gain fat in different places on your
	body\footnote{\href{https://www.youtube.com/watch?v=GbqN2sj8XyY}{VitruvianPhysique on Body Fat Distribution on Youtube}}\footnote{\href{https://www.youtube.com/watch?v=X_GeSVbAU3U}{VitruvianPhysique on Body Fat Distribution Part 2 on Youtube}}. 
	I don't know much more about it, but it seems to be purely genetic. The important thing to remember here is that you cannot control how your body stores or burns fat and from
	which areas. This is also what people mean when they say that losing fat is not site specific, you cannot burn fat from your belly just by doing crunches. Your body will decide
	where it takes the fat from based on the receptors I previously mentioned.
	
	Your body also has a carbohydrate storage just like it has for fat. This is used as an emergency source of energy, since the body can process carbs faster than fat, 
	for example if you come face to face with a lion and you have to run for your life, the carb storage will be used instead of the fat on your body. 
	Usually high intensity activities, such as sprinting really fast or lifting heavy weights for a short amount of time (anaerobic exercise) will make use of the carb stores.
	Just because you use carbs instead of fat it doesn't mean you will not lose weight if you are dieting. These carb stores are not permanent like fat, they get replenished
	every day (I usually associate this with computer memory --- carb stores are like volatile memory or RAM while fat stores are like disk memory). So if you burn 100kcal of
	carbs in a high intensity activity, your body will take 100kcal of carbs from your next meal to replenish these stores, so your meal will have 100kcal less that can be
	stored as fat. At the end of the day, it's calories in and calories out that matters. Most of the carb stores are located in muscles and it's called  muscle glycogen. 
	You also have them in liver as liver glycogen and a little bit in blood as plasma glucose. An average human adult will have around 503g of carbs (2012kcal) stored in body:
	around 400g in muscle glycogen, 100g in liver glycogen and 3g as plasma glucose.  
	
	Glycogen is important to understand random weight fluctuations, because it holds water. If you suddenly stop eating carbs (so the keto diet I previously mentioned), you will
	completely deplete your body of glycogen. This will result in a sudden loss of around 2kg of water from your body, which has nothing to do with you losing fat. If one day you
	have more carbs than the previous day then it's highly likely that your weight will go up the next day just because you have more glycogen and more water in your body, and not
	because you suddenly gained fat. It's a thing to keep in mind while you are dieting, and not get scared when you see sudden jumps in weight. Glycogen manipulation is something
	bodybuilders do before a contest too (peak week), to make themselves look more muscular, but more on this later.
		
	\section{Bulking \& Cutting}	
	
	Your TDEE is the total amount of energy your body needs every day to be able to function properly. If you eat less than that, your body has to go to fat stores or muscles to get that energy. If you eat more, the excess food will be stored on your body as fat or muscle. This is the reason people say that you need to be at a surplus to gain muscles, if you are at a caloric deficit then the proteins that should be stored on your muscles will be used for heat and movement instead. This phase in which you eat more than you should to build muscles is called bulking. Being at a caloric surplus will make it impossible for you to burn fat however, in fact you might end up gaining fat mass as well since the body is not optimal at storing pure muscles. To get rid of the excess fat you need to put yourself at a caloric deficit when you try to burn the fat stores while keeping the muscles you gained. This phase is called cutting. Bulking and cutting are normal cycles for bodybuilders, they will bulk, cut, bulk, cut and so on. There are instances when you can build muscles and cut down fat at the same time, I experienced this on myself when I started training again after a 3 months absence. I've read that it can also happen when you start training for the first time or if you take steroids, but more on steroids later. Building muscles and losing fat at the same time is also referred to as body recomposition\footnote{\href{https://www.healthline.com/nutrition/body-recomposition\#how-it-works}{Healthline Article on Body Recomposition}}.
	
	If you keep training (e.g. lifting weights) while cutting, your body will value your muscles more than the fat, so it will rather use fat to get the extra energy it needs. This is an oversimplified explanation of how everything works, in reality it's more complex than this. A different explanation would be this: the body has workers that take macronutrients (carbs, fat \& proteins) from food to places where they need to be. In this case the workers that take proteins to muscle is the testosterone in your body. Testosterone will race other workers that take macronutrients to produce heat and convert them to movement --- these workers prefer carbs and fat over protein. Based on how much testosterone you have and how easy it is for your body to rebuild the muscles you might be able to use all the proteins you get to gain muscles while the body will have to go to fat stores to get all the energy, so basically building muscles and losing fat at the same time. I've seen this analogy of testosterone with construction workers used a lot in fitness. 
	
	Knowing your TDEE, what is the caloric surplus you should aim for during a bulking phase? Most people seem to go 10-20\% more for what's considered a "lean bulk"\footnote{\href{https://www.youtube.com/watch?v=rCdba0UPTMk}{VitruvianPhysique on Lean Bulking on Youtube}}\footnote{\href{https://us.myprotein.com/thezone/nutrition/the-lean-bulk-how-to-minimize-fat-gain-while-bulking/}{Lean Bulk on Myprotein website}}\footnote{\href{https://www.youtube.com/watch?v=Ci3qXtNFU_w}{Mike Thurston on Lean Bulking on Youtube}}. For example if your TDEE is 2,500 calories a day, you should eat 2,750 - 3,000 calories to lean bulk. If you go more than that it's considered a "dirty bulk" in which you gain more fat than you should. This happens because there is a limit to how much muscle you can gain. Eating more might make you feel stronger, however strength and muscle growth are not perfectly correlated, but more on this later. Doing a dirty bulk means you'll be spending more time to cut the fat afterwards. It can also make you feel like shit, eating this much and dieting for so long. Sure it might give you more energy in the gym and make you even stronger (it's interesting how strength and muscle size are not perfectly correlated but more on this later) but it all depends on what you want to achieve. If you want to stay in shape all of the time then it's probably not for you, since the only time you'll be in shape is at the end of a cut for a short period of time. A lot of people speak against dirty bulks\footnote{\href{https://www.youtube.com/watch?v=DjEnkzhz5T4}{Greg Doucette on Bulking and Cutting on Youtube}}\footnote{\href{https://www.youtube.com/watch?v=xl0ZNFcvuJI}{Greg Doucette on Bulking on Youtube}} and I don't approve of them myself since I tried it once --- I felt like shit, out of shape and it made no sense to me since losing fat was one of the main reasons I got into this. You might also experience stretch marks on dirty bulks if you gain weight too fast\footnote{\href{https://www.bodybuilding.com/content/stretchmark-maintenance.html}{Article on Stretchmarks on Bodybuilding.com}}. One mistake I notice people make when bulking is that they fall into thinking that overeating junk food is fine, since they are bulking anyway. I made this mistake myself as well. You should still "diet" and track your calories and weight accordingly, and have cheat days if you really feel like eating junk food.
	
	How much should you eat when cutting down? The same holds true as for bulking, you should aim to eat 10-20\% less than you need every day. For a TDEE of 2,500 calories you should eat 2,000 - 2,250 calories.
	If you lose weight too fast there is the risk of losing muscles too, and you don't want this to happen since you worked so hard to gain them in the first place. 
	What many consider a good and safe pace to lose weight is 0.45-0.9kg per week (or 1-2 pounds)\footnote{\href{https://www.healthline.com/nutrition/losing-weight-too-fast}{Losing Weight too Fast on Healthline}}. The idea is that if you eat too little, your body will go into shock and try to use anything it can to keep you alive, including
	muscle stores. Transitioning from a bulk to a cut (and the other way around) should always happen gradually, to give your body time to accommodate\footnote{\href{https://www.ironbuiltfitness.com/transition-from-cutting-to-bulking}{Transitioning from Bulking to Cutting on ironbuiltfitness.com}}. 
	I noticed this on myself, I increased my caloric intake by more than 1,000 all of a sudden and I gained quite a bit of fat in a short amount of time. I would
	personally go back to maintenance and stay there for 1-2 weeks before starting to lean bulk.
	
	Now that you know about bulking \& cutting phase, how long do you do each and how do you plan them? For example, you could be bulking for 4 months and cutting for 1 month.
	However, there is no standard here, it depends on a lot of factors --- what type of bulk you did, how easy you gain fat, what type of cut you are going to do. A common
	approach is to pick your ideal body fat \% (from experience or if you can't, just estimate it looking at pictures) and try to stay close to that value when you bulk \& cut.
	For example my ideal body fat \% is 12\%. I want to end a bulk at +5\% of that value, so 17\%, and a cut at -5\% so 7\% body fat. Since single digit body fat is not that 
	amazing I'm happy to end it around 10\% body fat. If you never lifted before I would recommend starting with a bulk (to gain muscles, increase your metabolism and train your
	body to burn fat easier) for 3-4 months. After this bulking phase, I would cut for however long it takes me to get down to 10\% (could be 1 month or even 5+). After my cutting
	phase is over I would switch again to bulking until I hit 17-20\% then cut down again to 10\% and so on. Another thing to keep in mind is that you might want to end a cutting
	phase in a certain time of the year, for example before summer and then maybe maintain it over summer. This is also possible, you're in charge of however you want to split your
	bulking, cutting and maintenance phases. I would normally bulk over winter, like most people do and start my cutting phase in time to get down to 10\% for the summer. I usually
	start my cut beginning of March to reach my goal by June. A more detailed explanation: let's say I am at 80kg at an estimated 17\% body fat --- this means 66.4kg lean mass and
	13.6kg of fat. To get down to 10\% keeping my lean mass means I have to go down to 73.8kg (66.4kg lean mass and 7.4kg of fat), this means I need to lose 6.2kg of fat. I know I 
	feel comfortable with a diet where I lose 0.7kg per week. Doing the math, this means I need to diet for 8.9 weeks. Knowing that it gets harder to lose fat as you drop body fat
	\% and that my cheat days will set me back quite a bit, I think 12 weeks is a safe bet for me to reach my goal physique, this means I have to start my cutting phase
	start of March to get in shape by June! Even if I don't hit my goal of 10\%, I will be in pretty damn good shape for the summer.

	\section{Metabolism and Adaptive Thermogenesis}
	
	When people say metabolism they usually talk about BMR, which is the amount of energy you need in a day without extra physical activity. I discussed BMR previously and how you
	can compute it. Normally it should be constant for each individual. However, if you eat less than you should for a long period of time, studies showed that it can actually go
	down and not just because you drop body fat \%. For example, your body will produce less heat which is called adaptive
	thermogenesis\footnote{\href{https://www.ncbi.nlm.nih.gov/pmc/articles/PMC3673773/}{Adaptive Thermogenesis in Humans Article}}. Some people also call this metabolic damage and
	it seems this damage is not permanent and you can increase back your BMR after you go back to maintenance caloric 
	intake\footnote{\href{https://www.youtube.com/watch?v=bLboowVr2DM}{VitruvianPhysique on Metabolic Damage on Youtube}}. Apart from changes in BMR, your body will also burn
	less calories for moving around, since you lose mass which means less kinetic energy. There is also less calories burned from eating less, since it takes a bit of energy
	to break down food into macros. This is called the thermic effect of food and I will discuss it in more details later.
	
	So it's clear that while you are cutting down, your maintenance caloric intake will also drop (the same is true when you bulk, it goes up). How can you lose weight then? Well,
	the drop in caloric maintenance value (people also call this set point) will drop really slow, giving you time to lose fat and 
	
	However I don't think it's a good idea to keep lowering your caloric intake too much. Instead you should compensate with extra cardio to keep dropping weight.
	
	People saying they have fast/slow metabolism = bullshit.	
	

	
		
\end{document}
